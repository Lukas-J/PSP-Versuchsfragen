\documentclass[12pt,a4paper]{article}
\usepackage[top=3cm, bottom=4.5cm, left=2.5cm, right=2.5cm]{geometry}
\usepackage[ngerman]{babel}

\title{Versuch 5 Fragen}
\author{Marko, Felix, Steffen}

\begin{document}
\maketitle

\begin{itemize}
\item Erkläre os_yield() / wie implementiert?
Wieso gehen wir bei os_yield() aus allen kritischen Bereichen vor dem Interrupt?
- damit nach dem Aufruf des Interrupts nicht dauerhaft der gewählte Prozess durchläuft

\item Wo wurde os_yield verwendet? 
- shared memory

\item Kommen wir zu shared memory:
Wie sieht deine neue nibble Maske für welche zustände aus? 

\item Welche Hilfs Funktionen hast du geschrieben.
- os_sh_readOpen
- os_sh_writeOpen
- os_sh_close

\item Was machen sie?
- Bit masken umschreiben (dazu genau erklären was zu was)

\item Yielden sie nur einmal? 
- nein, yield() in einer while schlefie

\item Wo verwendet ihr critical section/in welchen Funktionen?
- os_sh_readOpen
- os_sh_writeOpen
- os_sh_close

\item Warum nur dort?
- damit während des lesens oder schreibens weitere andere Processe arbeiten

\item Brauchen wir bitmasken, wenn die shared memory nur ein byte groß ist?
- nein, weil ein byte operationen Atomar sind

\item Welche attribute hat dir Funktion os_sh_write oder os_sh_read?
- heap, ptr, ???, offset, length

\item Was macht der offset? Wo wird er eingesetzt?
- offset wird auf shared memory angewandt

\item Wieviele queues.
- 4

\item Welche timeslices?
- 1, 2, 4 ,8

\item Höchste prio hat welche timeslices?
- 1

\item Wie bestimmen wir, welcher process in welcher queue ist.
- mit den 2 MSB der Prio

\item Wie bestimmen wir, welchen Prozess wir aus einer queue nehmen?
- mit queue [tail]

\item In der mlfq, es befindet sich 1 prozess in der ersten queue. Wie gehen wir vor?
- timeslice verringen
- und abhängig von der queue woanders einfügen

\item Und wenn er blocked ist?
- nach hinten in die queue schieben, und nächsten prüfen

\item Wieviele Prozesse können gleichzeitig blocked sein?
- 1, da nach jeden osyield wieder ISR aufgerufen wird und dort alle blocked auf nicht blocked gesetzt werden
\end{itemize}

Guter Link mit toller Zusammenfassung des Versuchs:

https://docs.google.com/document/d/1hKJPj7zuzHpVJaf7l3_A3NGMaAHzxF6JSqp8GAGAWh0/edit
\end{document}