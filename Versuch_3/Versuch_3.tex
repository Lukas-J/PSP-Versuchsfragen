\documentclass[12pt]{article}
\usepackage[utf8]{inputenc}
\title{Versuch 2 Fragen}
\author{Lukas, Felix}

\begin{document}
\maketitle

\begin{itemize}
    \item Welche Speicherstruktur hat der ATMega 644? 
    \item Erkläre die Unterschiede zwischen SRAM und ROM.
    \item Erkläre den SRAM. Ab welcher Adresse wird addressiert? Was liegt davor?
    \item Wie geht das genau mit dem Sicherheitsabstand. Warum brauchen wir ihn? Welchen habt ihr gewählt?
    \item Was ist die Motivation für einen Stack und einen Heap?
    \item Warum brauchen wir einen Heap, wenn wir schon einen Stack haben?
    \item Erkläre das Protokoll mit der Allokationstabelle und den Nutzdaten.
    \item Erkläre das Schichtenmodell für die Speicherverwaltung.
    \item Erkläre free und malloc.
    \item Muss man die Allokationstabelle initialisieren? Wann macht man das?
    \item Muss man die Nutzdaten initialisieren?
    \item Wie wird der Speicher addressiert?
    \item Warum darf der Leerlaufprozess nicht terminieren?
    \item Wofür brauchen wir den dispatcher.
    \item Erkläre den dispatcher und kill.
    \item Was geschieht, wenn ein Prozess terminiert, der sich in einer kritischen Sektion befindet?
    \item Was passiert, wenn ein Prozess os_kill  mit seiner eigenen PID aufruft?
    \item Was passiert konkret, wenn wir in os_kill nicht in die Schleife (die auf einen neuen Prozess vom Scheduler wartet) gehen?
\end{itemize}

\end{document}
