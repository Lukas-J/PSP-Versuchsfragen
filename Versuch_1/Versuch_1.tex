\documentclass[12pt]{article}
\usepackage[utf8]{inputenc}
\title{Versuch 1 Fragen}
\author{
        Vaargk
}

\begin{document}
\maketitle
Wichtig ist es zu wissen welche Register der MC hat. Dazu gehören ja DDR, PORT und PIN. Auch wann man welches Register verwendet und ihre Namen zu können ist sinnvoll. Was muss man um eine LED zu schalten tun? Was muss man machen, damit ein Button funktioniert?
Zu dem ersten Versuchsteil wird gefragt, was auf der zusätzlichen Platine vorhanden ist. Was ist ein Komperator? Was gibt dieser aus? 1/0 Wann genau? Wie viele anschlüsse hat das R2R Netzwerk? Wie viele Möglichkeiten zur Belegung gibt es? 2⁸. Zudem welche Möglichkeiten der DA Wandlung gibt es? Wie viele Schritte muss ein Tracking Wandler im worst case machen? Wie viele muss ein SAR Wandler machen? Wann ist es sinnvoll einen SAR, wann einen Tracking Wandler zu verwenden? Wenn es ein Sägezahnsignal gibt, welchen Wandler sollte man verwenden? Ist der Auf der Platine gegebene Wandler besser als der Wandler auf dem externen Bord? Was sind die wichtigen Angaben an den Digital/analog Wandler? (Auflösung und Abtastrate) Was ist Quantisierung? Was passiert beim Alasing effekt? wie schnell sollte die Abtastrate sein, damit es nicht zu dem Effekt kommt?
\end{document}

