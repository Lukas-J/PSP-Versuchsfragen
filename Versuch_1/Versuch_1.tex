\documentclass[12pt]{article}
\usepackage[utf8]{inputenc}
\title{Versuch 1 Fragen}
\author{
        Vaargk
}

\begin{document}
\maketitle
\begin{itemize}
    \item Wichtig ist es zu wissen welche Register der MC hat. Dazu gehören ja DDR, PORT und PIN.
    \item Auch wann man welches Register verwendet und ihre Namen zu können ist sinnvoll. Was muss man um eine LED zu schalten tun? Was muss man machen, damit ein Button funktioniert?
    \item Zu dem ersten Versuchsteil wird gefragt, was auf der zusätzlichen Platine vorhanden ist.
    \item Was ist ein Komperator? Was gibt dieser aus? 1/0 Wann genau?
    \item Wie viele Anschlüsse hat das R2R Netzwerk? Wie viele Möglichkeiten zur Belegung gibt es? $2^8$
    \item Welche Möglichkeiten der DA Wandlung gibt es? 
    \item Wie viele Schritte muss ein Tracking Wandler im worst case machen?
    \item Wie viele Schritte muss ein SAR Wandler machen?
    \item Wann ist es sinnvoll einen SAR, wann einen Tracking Wandler zu verwenden?
    \item Wenn es ein Sägezahnsignal gibt, welchen Wandler sollte man verwenden?
    \item Ist der Auf der Platine gegebene Wandler besser als der Wandler auf dem externen Bord?
    \item Was sind die wichtigen Angaben an den Digital/analog Wandler? (Auflösung und Abtastrate)
    \item Was ist Quantisierung?
    \item Was passiert beim aliasing effekt? Wie schnell sollte die Abtastrate sein, damit es nicht zu dem Effekt kommt?
\end{itemize}
\end{document}

