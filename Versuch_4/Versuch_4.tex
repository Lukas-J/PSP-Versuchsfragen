\documentclass[12pt,a4paper]{article}
\usepackage[top=3cm, bottom=4.5cm, left=2.5cm, right=2.5cm]{geometry}
\usepackage[ngerman]{babel}

\title{Versuch 4 Fragen}
\author{Hauke, Tom}

\begin{document}
\maketitle

\begin{itemize}
    \item Was habt ihr für den Versuch 4 implementiert?
    \item Was habt ihr für die Initioalisierung des externen Speichers genau gemacht?
    \item Welche Register habt ihr gesetzt? Was wurde in jedem Register gesetzt? (Status/Control/Data Register, z.B SPIE im Controlregister) 
    \item Was für Ein/Ausgänge hat der externe Speicher? (Achtung: CS low active)
    \item Was für eine Geschwindigkeit habt ihr für den externen Speicher gewählt?
    \item Was ist die schnellste Allokationsstrategie? (Next Fit) Was ist die beste? (Situationsabhängig)
    \item Was für eine Größe hatte der externe Speicher? Was für eine Größe nutzt ihr und warum genau diese? Warum muss die Größe durch drei teilbar sein?
    \item Wie funktioniert das Schreiben/Lesen im externen Speicher genau?\\ Was für Rückgabewerte und Parameter werden verwendet?
    \item Was darf auf keinen Fall beim Lesen und Schreiben passieren? (Scheduler Prozesswechsel)
    \item Was für Situationen gibt es beim Realloc? 
    \item In welchen Situtaionen müssen beim Realloc Daten kopiert werden? 
    \item Wie konntet ihr freeProcessMemory verschnellern? (Ramen mit Erklärung der Funktionsweise)
\end{itemize}
\end{document}
